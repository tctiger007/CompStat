From the above result, we are able to restore the original image quite well. In particular, $\sigma=5$ works the best, and first-order neighborhood is slightly better than second-order. 
The initial $\mathbf{x}^{(0)}=57.5$ works similar as $\mathbf{x}^{(0)}$ as the observed data. The mean image in every case is pretty smooth since it is averaged over 100 images.
$\sigma$ seems to be a more important factor to the sampled images compared with neighborhood structure.
In the conditional posterior distribution, variance is equal to $\frac{\sigma^2}{v_i+1}$.
When $\sigma$ changes, the variance changes in the order of $\sigma^2$ (when the neighborhood structure is fixed). 
But when neighborhood structure changes and $\sigma$ is fixed, the variance only changes by a factor of $(\frac{1}{5}\sim\frac{1}{9})$.
Although the mean of the conditional posterior also changes when neighborhood structure changes, but it depends on the data, and the order of the difference between mean is only a linear combination of $y_i$ and $\bar{x}_{\delta_i}$. In this sense, $\sigma$ is a more significant factor. Our results have also suggested this idea. 
There is not a significant difference between Gibbs saplers with different initial starting image $\mathbf{x}^{(0)}$, indicating the Markov chain moves fast from the initial $\mathbf{x}^{(0)}$ and converges to roughly the same result after some iterations. 