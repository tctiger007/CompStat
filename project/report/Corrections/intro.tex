Markov chain Monte Carlo (MCMC) has become a popular method in solving complicated statistical problems.
MCMC is particularly useful in Bayesian analysis, where complex and high dimensional integrals are involved in obtaining posterior distributions \cite{Cowles1996}. 
One application of MCMC is Bayesian analysis of Markov random field (MRF), which is considered as a set of random variables that have Markov property. 
A MRF can be regular or irregular, and based on the individual sites and associated random variables, it can be further classified into different groups \cite{Besag1993}.
MRF is extensively used to model studies in imaging processing.

Bayesian image analysis includes the removal of noise of an image, the generation of better image from an original noisy image, the restoration of multi-dimensional images from lower dimensional images and etc. 
Besag \textit{et al.} pointed out the often neglected area in Bayesian image analysis that is indeed useful and important -- image restoration \cite{Besag1991}. 
Our project explores the application of this particular practical importance. 

Bayesian image analysis uses probability models to incorporate the specific prior knowledge into a defined image. 
Hammersley and Clifford Theorem \cite{Hammersley1971}, first proved in an unpublished work by John Hammersley and Peter Clifford in 1971, is the key to construct proper model through the conditional probability distribution in MRF-related image restoration problems.
Using this theorem, we can specify the joint distribution of pixels to a normalizing constant \cite{Givens}.

There are a variety of schemes for lattice systems, among which binary, Gaussian, auto-binomial, auto-Poisson, and auto-exponential schemes are quite common. 
Julian Besag have done extensive studies in the field, and more detailed review of each of the case can be read in \cite{Besag1993}.
Our project is a Gaussian scheme, which models the joint distribution to a multivariate normal distribution.

The central problem in MRF is that there is no direct method to simulate general multivariate distributions \cite{Besag1993}, so algorithm that is based on the corresponding univariate local information is quite helpful. 
Our project deals with one of the easiest scenarios in MRF, where the lattice is regular and the pixel image is as small as 20 $\times$ 20, which in real life is unrealistic to encounter. 
However, even with this small number of pixels, it could already create an enormous number of terms in the conditional probability calculation. 
Hence, MCMC can be used in this case to overcome the difficulty of computational burden. 

In particular, Gibbs sampling, one of the MCMC tools which updates the component according to the current local information ensures the convergence to the joint distribution under general conditions \cite{Besag1993}, and therefore is widely used in solving MRF-related problems. 
Sampling from multivariate densities in Gibbs sampling is pretty straigtforward as long as we can derive the conditional probability densities.
Generally, we first treat the other variables as fixed in the joint probability densities, then we investigate how to sample from the conditional distribution.

In our project, the prior and likelihood are given.
The prior distribution of the observed data is a Gaussian distribution.
We apply Gibbs sampling to generate a collection of images from the posterior distribution with known prior density  and likelihood  from a 20 $\times$ 20 pixel image.
We also compare the images generated from Gibbs samplers with differing $\mathbf{x}^{(0)}$, $\sigma$, and neighborhood structure. 